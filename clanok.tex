\documentclass[]{article}

\usepackage[utf8]{inputenc}
\usepackage{graphicx}
\usepackage{url}


%\usepackage{cite}

\pagestyle{headings}

\title{Evolution of recommendation algorithms}

\author{Danyil Pastoshchuk\\[2pt]
	{\small Slovenská technická univerzita v Bratislave}\\
	{\small Fakulta informatiky a informačných technológií}\\
	{\small \texttt{xpastoshchuk@stuba.sk}}
}

\date{\small 24 september 2024}

\begin{document}
	
	

\maketitle
\begin{abstract}
In today’s day and age, recommendation systems play an integral role in directing users’ attention to products or content they might want to consume. They have become ubiquitous in areas like e-commerce, entertainment, news, social media, marketing, and find occasional use in education, finance and healthcare. With the digitalization of services it is becoming exceedingly harder to navigate the overwhelming variety of products each of them has to offer, which necessitates their continuous improvement. Another motivating factor in their development is the striving of websites to keep the user engaged with their content for as long as possible by providing products they are more likely to have interest in, thus maximizing profits. Modern systems work by analyzing data on user behavior and preferences, identifying patterns in it and making predictions as to what the user might like next. The primary task of this article is to briefly showcase the progression of recommendation systems in chronological order: from the conception of the first, rather primitive by present-day standards algorithms, to the newest state-of-the-art AI-based developments that have revolutionalized the way we consume information. We will introduce some of the classic techniques  that were widely used before the advent of deep learning, and see how the same basic ideas are combined with it and modified to give a much fruitful result.\cite{r1}\cite{r2}

\end{abstract}

\bibliography{literatura}
\bibliographystyle{plain}








\end{document}